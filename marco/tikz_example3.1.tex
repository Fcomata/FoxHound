% Listas (Prueba con iteracion)
\documentclass{article}
\usepackage{tikz}
\usepackage{ifthen}
\usetikzlibrary{matrix}
\begin{document}
\begin{tikzpicture}
\newcounter{temp}
%Declaracion de la matriz y sus propiedades
  \matrix (m) [matrix of math nodes,row sep=3mm,column sep=-0mm,minimum width=2mm, rectangle]

\forloop{temp}{1}{\value{temp} < 5}{
	\node[draw,rectangle, fill=blue!20]{F_x(\omega)};
}


}

%Creacion de los nodos de la matriz
%  {
%     \node[draw,rectangle, fill=blue!20]{F_x(\omega)}; &
%     \node[draw,rectangle, fill=blue!20]{F_y(\omega)}; &    
%     \node[draw,rectangle, fill=blue!20]{F_z(\omega)};\\}; 

%Comando para crear un nodo sobre otro
%  \path[-stealth]
%  (m-1-2) node [draw,rectangle, fill=blue!20]{$F_z(\omega)$}
;
\end{tikzpicture}
\end{document}
